\section{Air wall, Open air connection between zones}\label{air-wall-open-air-connection-between-zones}

It is extremely difficult to model the interactions between thermal zones which are connected by a large opening. If the zones are controlled to the same conditions, then there is little to be gained by making them interact, so you could neglect any connections between the zones. In fact, if this is the case, you might consider combining the spaces into a single thermal zone. If you expect the zones to have significantly different temperatures and/or humidities, then use one of the following options. If they are modeled as separate zones, EnergyPlus models only what is explicitly described in the input file, so simply leaving a void (no surfaces) between two zones will accomplish nothing - the two zones will not be connected. The main interactions which occur across the dividing line between two zones which are fully open to each other are:

1)~~~Convection or airflow, which will transfer both sensible heat and moisture. Some modelers use MIXING (one-way flow) or CROSS MIXING (two-way flow) to move air between the zones, but the user must specify airflow rates and schedules for this flow, and it cannot be automatically linked to HVAC system operation. Other modelers use AirFlowNetwork with large vertical openings between the zones as well as other openings and cracks in the exterior envelope to provide the driving forces. It can also be connected with the HVAC system (for limited system types). This requires a much higher level of detailed input and should be used only if the detailed specification data is available. If the two zones are controlled to similar conditions, this effect could be safely neglected.

2)~~~Solar gains and daylighting. The only way to pass solar and daylight from one zone to the next is through a window or glass door described as a subsurface on an interzone wall surface. Note that all solar is diffuse after passing through an interior window.

3)~~~Radiant (long-wave thermal) transfer. There is currently no direct radiant exchange between surfaces in different thermal zones. Windows in EnergyPlus are opaque to direct radiant exchange, so an interzone window will not behave any differently than an opaque interzone surface for this aspect. However, a large interzone surface (opaque or window) would result in some indirect radiant exchange since the interzone surface will exchange directly with surfaces in zone A and in zone B. The surface thermal resistance should be low in order to most closely approximate this effect.

4)~~~Conduction. If an interzone surface is placed between the two zones, it will conduct sensible heat between the two zones. Using a low thermal resistance helps to move radiant exchange between the zones.

5)~~~~Visible and thermal radiant output from internal gains. These gains will not cross zone boundaries. But again, they will impact any interzone surfaces, so some of the energy may move across to the next zone."

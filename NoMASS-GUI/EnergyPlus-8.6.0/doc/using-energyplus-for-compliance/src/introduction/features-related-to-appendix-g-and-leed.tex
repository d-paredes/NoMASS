\section{Features Related to Appendix G and LEED}\label{features-related-to-appendix-g-and-leed}

EnergyPlus has several built-in features to make demonstrating compliance with ASHRAE Standard 90.1 Appendix G easier. Appendix G is used in Energy and Atmosphere Credit 1 Optimize Energy Performance of the U.S. Green Building Council (USGBC) building certification system called Leadership in Energy and Environmental Design (LEED) Green Building Rating System™. The following sections describe the features.

\subsection{Baseline Building Rotations}\label{baseline-building-rotations}

Table G3.1 Section 5 Building Envelope of ASHRAE Standard 90.1 Appendix G requires that the baseline building be simulated facing four directions:

Orientation. The baseline building performance shall be generated by simulating the building with its actual orientation and again after rotating the entire building 90, 180, and 270 degrees, then averaging the results.

This provision intends to provide a baseline that is neutral to building orientation so that buildings purposely oriented to minimize energy use can realize a percent savings.

\textbf{Step 1.} The first step of performing the building rotations is to include a Compliance:Building object which contains the Building Rotation for Appendix G field. The value of the field should be different for four simulations, 0, 90, 180 and 270 (see Step 2 below). This is further documented in the Input Output Reference.

\begin{lstlisting}

Compliance:Building,
      90;                 Building Rotation for Appendix G
\end{lstlisting}

Also make sure that the HTML summary reports will be generated by including the following:

\begin{lstlisting}

OutputControl:Table:Style,
    HTML;                      !- type

  Output:Table:SummaryReports,
    AllSummary;                !- type
\end{lstlisting}

\textbf{Step 2.} The ParametricPreprocessor automates the creation of the four simulation input files so that only a single simulation file is needs to be created by the user. By setting the file name suffixes (which are appended to the existing file name for each simulation), setting a variable that varies for the four simulations, and setting the Building Rotation for Appendix G to this variable, the user can automatically create the four files when using EnergyPlus. The following example shows what should be included in the EnergyPlus file. The Parametric objects are described further in the Input Output Reference and the ParametricPreprocessor is described in the Auxiliary Programs documentation.

\begin{lstlisting}
Parametric:FileNameSuffix,
  Names,
  G000,
  G090,
  G180,
  G270;
\end{lstlisting}

\begin{lstlisting}
Parametric:SetValueForRun,
  $appGAngle,            !- Parameter Name
  0.0,                   !- Value 1
  90.0,                  !- Value 2
  180.0,                 !- Value 3
  270.0;                 !- Value 4
\end{lstlisting}

\begin{lstlisting}
Compliance:Building,
    = $appGAngle;        !- Building Rotation for Appendix G {deg}
\end{lstlisting}

\begin{lstlisting}
OutputControl:Table:Style,
  HTML;                      !- type
\end{lstlisting}

\begin{lstlisting}
Output:Table:SummaryReports,
  AllSummary;                !- type
\end{lstlisting}

\textbf{Step 3.} Run the simulation using EP-Launch. Using the ``Single Input File'' tab of EP-Launch select the input that contains the Compliance:Building, Parametric:FileNameSuffix, Parametric:SetValueForRun objects as shown above in Step 2 as well as the weather file. In EP-Launch, make sure under VIEW .. OPTIONS .. MISCELLANEOUS that the RUN PARAMETRICPREPROCESSOR option is checked.

\textbf{Step 4}. Review the results and revise the model inputs as needed. You can view the results of the simulation by using the EP-Launch and the History tab. At the bottom of the history list, four simulations that use the G000, G090, G180 and G270 file name suffixes should appear and the result files associated with each can be selected and opened.

\textbf{Step 5.} Use the AppGPostProcess to average results across the simulations. In EP-Launch under the UTILITIES tab, select AppGPostProcess utility and select one of the HTML files resulting from the multiple simulation runs (e.g., \textless{}filename\textgreater{}-G000.html). This will open all four files and generate a new output file with the file suffix GAVG for the HTML and other results files that can be opened on that tab.

\subsection{Completing LEED Forms from Tabular Reports}\label{completing-leed-forms-from-tabular-reports}

The U.S. Green Building Council building certification system called Leadership in Energy and Environmental Design (LEED) Green Building Rating System™ includes Energy and Atmosphere Credit 1 Optimize Energy Performance. Credit 1 includes an option that requires a series of building energy simulations that follow the procedures of ASHRAE Standard 90.1 Appendix G Performance Rating Method. The~ LEED Summary report provides many of the simulation results required for the forms. The report can be produced by specifying LEEDSummary in Output:Table:SummaryReports which is also part of the AllSummary option. 

The EAp2-4/5 Performance Rating Method Compliance subtable has some special options in order to show end uses that are different than the end-uses that are typically used with EnergyPlus. By specifing the end-use subcategories of ``Cooking" most commonly with the ElectricEquipment and GasEquipment objects, ``Elevators and Escalators" most commonly with the ElectricEquipment object, ``Industrial Process" most commonly with the ElectricEquipment and GasEquipment objects, ``Interior Lighting-Process" most commonly with the Lights object, or ``Fans-Parking Garage" most commonly with the Exterior:FuelEquipment object, those end uses are shown in the EAp2-4/5 Performance Rating Method Compliance subtable.


Report: \textbf{LEED Summary}

For: \textbf{Entire Facility}

Timestamp: \textbf{2013-03-01 15:24:37}

\textbf{Sec1.1A-General Information}

\begin{longtable}[c]{>{\raggedright}p{2.21in}>{\raggedright}p{3.78in}}
\toprule 
 & Data \tabularnewline
\midrule
\endfirsthead

\toprule 
 & Data \tabularnewline
\midrule
\endhead

Heating Degree Days & 1748 \tabularnewline
Cooling Degree Days & 506 \tabularnewline
Climate Zone & 5A \tabularnewline
Weather File & Chicago Ohare Intl Ap IL USA TMY3 WMO\#=725300 \tabularnewline
HDD and CDD data source & Weather File Stat \tabularnewline
Total gross floor area [m2] & 927.20 \tabularnewline
Principal Heating Source & Natural Gas \tabularnewline
\bottomrule
\end{longtable}

\textbf{EAp2-1. Space Usage Type}

\begin{longtable}[c]{>{\raggedright}p{1.2in}>{\raggedright}p{1.2in}>{\raggedright}p{1.2in}>{\raggedright}p{1.2in}>{\raggedright}p{1.2in}}
\toprule 
 & Space Area [m2] & Regularly Occupied Area [m2] & Unconditioned Area [m2] & Typical Hours/Week in Operation [hr/wk] \tabularnewline
\midrule
\endfirsthead

\toprule 
 & Space Area [m2] & Regularly Occupied Area [m2] & Unconditioned Area [m2] & Typical Hours/Week in Operation [hr/wk] \tabularnewline
\midrule
\endhead

SPACE1-1 & 99.16 & 99.16 & 0.00 & 55.06 \tabularnewline
SPACE2-1 & 42.73 & 42.73 & 0.00 & 55.06 \tabularnewline
SPACE3-1 & 96.48 & 96.48 & 0.00 & 55.06 \tabularnewline
SPACE4-1 & 42.73 & 42.73 & 0.00 & 55.06 \tabularnewline
SPACE5-1 & 182.49 & 182.49 & 0.00 & 55.06 \tabularnewline
PLENUM-1 & 463.60 & 463.60 & 0.00 & 0.00 \tabularnewline
Totals & 927.20 & 927.20 & 0.00 & ~ \tabularnewline
\bottomrule
\end{longtable}

\textbf{EAp2-2. Advisory Messages}

\begin{longtable}[c]{@{}ll@{}}
\toprule 
 & Data \tabularnewline
\midrule
\endfirsthead

\toprule 
 & Data \tabularnewline
\midrule
\endhead

Number of hours heating loads not met & 0.00 \tabularnewline
Number of hours cooling loads not met & 10.75 \tabularnewline
Number of hours not met & 10.75 \tabularnewline
\bottomrule
\end{longtable}

\textbf{EAp2-3. Energy Type Summary}

\begin{longtable}[c]{>{\raggedright}p{1.2in}>{\raggedright}p{1.2in}>{\raggedright}p{1.2in}>{\raggedright}p{1.2in}>{\raggedright}p{1.2in}}
\toprule 
 & Utility Rate & Virtual Rate [\textbackslash\$/unit energy] & Units of Energy & Units of Demand \tabularnewline
\midrule
\endfirsthead

\toprule 
 & Utility Rate & Virtual Rate [\textbackslash\$/unit energy] & Units of Energy & Units of Demand \tabularnewline
\midrule
\endhead

Electricity & EXAMPLEA EXAMPLEI-SELL & 0.055 & kWh & kW \tabularnewline
Natural Gas & EXAMPLEA-GAS & 0.569 & Therm & Therm/Hr \tabularnewline
Other & ~ & ~ & ~ & ~ \tabularnewline
\bottomrule
\end{longtable}

\textbf{EAp2-4/5. Performance Rating Method Compliance}

\begin{longtable}[c]{>{\raggedright}p{0.85in}>{\raggedright}p{0.85in}>{\raggedright}p{0.85in}>{\raggedright}p{0.85in}>{\raggedright}p{0.85in}>{\raggedright}p{0.85in}>{\raggedright}p{0.85in}}
\toprule 
 & Electric Energy Use [GJ] & Electric Demand [W] & Natural Gas Energy Use [GJ] & Natural Gas Demand [W] & Other Energy Use [GJ] & Other Demand [W] \tabularnewline
\midrule
\endfirsthead

\toprule 
 & Electric Energy Use [GJ] & Electric Demand [W] & Natural Gas Energy Use [GJ] & Natural Gas Demand [W] & Other Energy Use [GJ] & Other Demand [W] \tabularnewline
\midrule
\endhead

Interior Lighting & 81.24 & 7125.00 & 0.00 & 0.00 & 0.00 & 0.00 \tabularnewline
Exterior Lighting & 0.00 & 0.00 & 0.00 & 0.00 & 0.00 & 0.00 \tabularnewline
Space Heating & 0.00 & 0.00 & 103.92 & 62499.99 & 0.00 & 0.00 \tabularnewline
Space Cooling & 17.63 & 9523.66 & 0.00 & 0.00 & 0.00 & 0.00 \tabularnewline
Pumps & 1.54 & 319.57 & 0.00 & 0.00 & 0.00 & 0.00 \tabularnewline
Heat Rejection & 0.00 & 0.00 & 0.00 & 0.00 & 0.00 & 0.00 \tabularnewline
Fans-Interior & 7.01 & 609.44 & 0.00 & 0.00 & 0.00 & 0.00 \tabularnewline
Fans-Parking Garage & 0.00 & 0.00 & 0.00 & 0.00 & 0.00 & 0.00 \tabularnewline
Service Water Heating & 0.00 & 0.00 & 0.00 & 0.00 & 0.00 & 0.00 \tabularnewline
Receptacle Equipment & 47.70 & 4500.00 & 0.00 & 0.00 & 0.00 & 0.00 \tabularnewline
Interior Lighting (process) & 0.00 & 0.00 & 0.00 & 0.00 & 0.00 & 0.00 \tabularnewline
Refrigeration Equipment & 0.00 & 0.00 & 0.00 & 0.00 & 0.00 & 0.00 \tabularnewline
Cooking & 0.00 & 0.00 & 0.00 & 0.00 & 0.00 & 0.00 \tabularnewline
Industrial Process & 0.00 & 0.00 & 0.00 & 0.00 & 0.00 & 0.00 \tabularnewline
Elevators and Escalators & 0.00 & 0.00 & 0.00 & 0.00 & 0.00 & 0.00 \tabularnewline
Total Line & 155.12 & ~ & 274.01 & ~ & 0.00 & ~ \tabularnewline
\bottomrule
\end{longtable}

\textbf{EAp2-6. Energy Use Summary}

\begin{longtable}[c]{@{}lll@{}}
\toprule 
 & Process Subtotal [GJ] & Total Energy Use [GJ] \tabularnewline
\midrule
\endfirsthead

\toprule 
 & Process Subtotal [GJ] & Total Energy Use [GJ] \tabularnewline
\midrule
\endhead

Electricity & 47.70 & 155.12 \tabularnewline
Natural Gas & 0.00 & 274.01 \tabularnewline
Other & 0.00 & 0.00 \tabularnewline
Total & 47.70 & 429.13 \tabularnewline
\bottomrule
\end{longtable}

\textbf{EAp2-7. Energy Cost Summary}

\begin{longtable}[c]{p{1.5in}p{2.21in}p{2.28in}}
\toprule 
 & Process Subtotal [\textbackslash\$] & Total Energy Cost [\textbackslash\$] \tabularnewline
\midrule
\endfirsthead

\toprule 
 & Process Subtotal [\textbackslash\$] & Total Energy Cost [\textbackslash\$] \tabularnewline
\midrule
\endhead

Electricity & 552.55 & 1796.99 \tabularnewline
Natural Gas & 0.00 & 1478.58 \tabularnewline
Other & 0.00 & 0.00 \tabularnewline
Total & 552.55 & 3275.57 \tabularnewline
\bottomrule
\end{longtable}

\emph{Process energy cost based on ratio of process to total energy.}

\textbf{L-1. Renewable Energy Source Summary}

\begin{longtable}[c]{@{}lll@{}}
\toprule 
 & Rated Capacity [kW] & Annual Energy Generated [GJ] \tabularnewline
\midrule
\endfirsthead

\toprule 
 & Rated Capacity [kW] & Annual Energy Generated [GJ] \tabularnewline
\midrule
\endhead

Photovoltaic & 0.00 & 0.00 \tabularnewline
Wind & 0.00 & 0.00 \tabularnewline
\bottomrule
\end{longtable}

\textbf{EAp2-17a. Energy Use Intensity - Electricity}

\begin{longtable}[c]{@{}ll@{}}
\toprule 
 & Electricty [MJ/m2] \tabularnewline
\midrule
\endfirsthead

\toprule 
 & Electricty [MJ/m2] \tabularnewline
\midrule
\endhead

Interior Lighting & 87.62 \tabularnewline
Space Heating & 0.00 \tabularnewline
Space Cooling & 19.02 \tabularnewline
Fans-Interior & 7.56 \tabularnewline
Service Water Heating & 0.00 \tabularnewline
Receptacle Equipment & 51.44 \tabularnewline
Miscellaneous & 1.66 \tabularnewline
Subtotal & 167.30 \tabularnewline
\bottomrule
\end{longtable}

\textbf{EAp2-17b. Energy Use Intensity - Natural Gas}

\begin{longtable}[c]{@{}ll@{}}
\toprule 
 & Natural Gas [MJ/m2] \tabularnewline
\midrule
\endfirsthead

\toprule 
 & Natural Gas [MJ/m2] \tabularnewline
\midrule
\endhead

Space Heating & 112.08 \tabularnewline
Service Water Heating & 0.00 \tabularnewline
Miscellaneous & 183.45 \tabularnewline
Subtotal & 295.53 \tabularnewline
\bottomrule
\end{longtable}

\textbf{EAp2-17c. Energy Use Intensity - Other}

\begin{longtable}[c]{@{}ll@{}}
\toprule 
 & Other [MJ/m2] \tabularnewline
\midrule
\endfirsthead

\toprule 
 & Other [MJ/m2] \tabularnewline
\midrule
\endhead

Miscellaneous & 0.00 \tabularnewline
Subtotal & 0.00 \tabularnewline
\bottomrule
\end{longtable}

\textbf{EAp2-18. End Use Percentage}

\begin{longtable}[c]{@{}ll@{}}
\toprule 
 & Percent [\%] \tabularnewline
\midrule
\endfirsthead

\toprule 
 & Percent [\%] \tabularnewline
\midrule
\endhead

Interior Lighting & 18.93 \tabularnewline
Space Heating & 24.22 \tabularnewline
Space Cooling & 4.11 \tabularnewline
Fans-Interior & 1.63 \tabularnewline
Service Water Heating & 0.00 \tabularnewline
Receptacle Equipment & 11.11 \tabularnewline
Miscellaneous & 39.99 \tabularnewline
\bottomrule
\end{longtable}

\subsubsection{LEED Form Section 1.4 -- Comparison of Proposed Design Versus Baseline Design}\label{leed-form-section-1.4-comparison-of-proposed-design-versus-baseline-design}

Unlike other portions of the LEED forms, Section 1.4 Comparison of Proposed Design Versus Baseline Design is less structured with many possible ``Model Input Parameters'' to match with the ``Proposed Design Input'' and ``Baseline Design Input''.

\emph{1. Exterior wall, underground wall, roof, floor, and slab assemblies including framing type, assembly R-values, assembly U-factors, and roof reflectivity when modeling cool roofs. (Refer to ASHRAE 90.1 Appendix A)}

Use the values from the Envelope Summary table, Opaque Exterior subtable for Construction, Reflectance, U-Factor with Film, and U-Factor no Film.

\emph{2. Fenestration types, assembly U-factors (including the impact of the frame on the assembly, SHGCs, and visual light transmittances, overall window-to-gross wall ratio, fixed shading devices, and automated movable shading devices}

Use the values from Input Verification and Results Summary table, Window-Wall Ratio table Window-Wall Ratio percentage as well as Skylight-Roof Ratio table. Also use the values from the Envelope Summary table, Fenestration subtable for Construction, Area of Openings, U-Factor, SHGC, Visible Transmittance, and Shade Control. For shading use the Shading Summary report including Sunlit Fraction and Window Control.

\emph{3. Interior lighting power densities, exterior lighting power, process lighting power, and lighting controls modeled for credit.}

Use the values from the Lighting Summary report for Interior Lighting, Daylighting and Exterior Lighting including Lighting Power Density, Schedule Name, Average Hours/Week, Daylighting Type, Control Type, Fraction Controlled, Lighting Installed in Zone, Lighting Controlled.

\emph{4. Receptacle equipment, elevators or escalators, refrigeration equipment and other process loads.}

Use the values from Input Verification and Results Summary table, Zone Summary subtable for Plug and Process.

\emph{5. HVAC system information including types and efficiencies, exhaust heat recovery, pump power and controls, and other pertinent system information. (Include the ASRHAE 90.1-2004 Table G3.1.1B Baseline System Number)}

Use the Equipment Summary report, Central Plant subtable for Type, Nominal Capacity and Nominal Efficiency; Cooling Coils subtable for Nominal Total Capacity and Nominal Efficiency; Heating Coils subtable for Type, Nominal Total Capacity and Nominal Efficiency; Fans subtable for Type, Total Efficiency, Delta Pressure, Max Flow Rate, Rated Power, Motor Heat in Air Fraction; Pumps subtable for Type, Control, Head, Power and Motor Efficiency. Use the System Summary report, Economizer subtable for High Limit Shutoff Control and Minimum Outdoor Air; the Demand Controlled Ventilation using Controller:MechanicalVentilation subtable.

\emph{6. Domestic hot water system type, efficiency and storage tank volume.}

Use the Equipment Summary report, Service Water Heating subtable for Type, Storage Volume, Input, Thermal Efficiency, Recovery Efficiency, and Energy Factor.

\emph{7. General schedule information.}

Use the Lighting Summary report, Interior Lighting subtable Schedule Name and Average Hours/Week; Exterior Lighting subtable Schedule Name and Average Hours/Week. Use the System Summary report, Demand Controlled Ventilation using Controller:MechanicalVentilation subtable for the Air Distribution Effectiveness Schedule.

\chapter{EnergyPlus Install Contents}\label{energyplus-install-contents}

EnergyPlus Interfaces will naturally need to access the installed EnergyPlus programs, library files, documentation.~ It will help to describe how EnergyPlus is installed (on a Windows™ computer).~ The EnergyPlus install is written using the WISE™ installation software.

The scheme of installing EnergyPlus includes a ``root'' directory/folder and all subsequent programs installed as part of the installation are child folders under the parent/root folder.~ Several optional components can be selected during install.

The basic (required) installation has crucial files installed in the parent folder -- these include \textbf{EnergyPlus.exe}, \textbf{Energy+.idd} (the input data dictionary), \textbf{EPMacro.exe}.~ The basic installation also includes a child folder ``DataSets'' that contains the EnergyPlus ``libraries''.~ As distributed, EnergyPlus includes several library files, formatted in the standard EnergyPlus IDF format.~ These include thermal material properties, moisture material properties, glass and other properties for windows, constructions (material sandwiches which describe walls, windows, roofs), fluid properties, locations, design day definitions, and basic schedule definitions.~ There, of course, may be additional data sets added as well as future datasets edited for selectable use from the EPMacro program.~ The Templates folder is also included in the basic install.~ Currently, the Templates folder contains documentation and the HVAC Template files.~ These files can be used somewhat ``automatically'' to produce HVAC loop structures for running with EnergyPlus.

Optional components of the EnergyPlus install: \textbf{Documentation}, \textbf{EP-Launch}, \textbf{IDFEditor}, \textbf{Sample} \textbf{Files}, \textbf{Weather} \textbf{Converter}, and other auxiliary type programs.~ All the components are selected by default; to not install them the user must ``un-select'' them individually.~ Highlights of several:

\textbf{Documentation:}~ The EnergyPlus package includes a comprehensive set of documents intended to help the user and others understand the EnergyPlus program, usage, and other appropriate information.~ All documents are created in PDF (Portable Document Format).~ There is ``index'' to all documents to make searching for a subject easier.~A shortcut to the Documentation folder and the ``main-menu'' document is included.~ The main-menu document is a navigation aid to the remainder of the documents.~ We may want to make the documentation a non-optional component.

\textbf{EP-Launch:~} The components of EP-Launch are installed in the parent directory (help files installed in the documentation directory).~ Because the basic EnergyPlus program runs as a console application, many beta users did not understand how to make the program execute.~ While the developers may be able to tailor the EnergyPlus executable to be more callable under the Windows™ platform, this is still a useful program.~ EP-Launch uses the EPL-Run.bat file and prepends several ``set'' commands that are used in the bat file.~ It creates the actual batch file for the run as ``RunEP.bat'' and then calls the operating system to execute the file.~~ Having the EPL-Run.bat file as external to the EP-Launch program means that others may tailor the batch file more appropriately to how things are run though this may not be preserved with a future EnergyPlus install. EP-Launch can also execute several utility programs from the ``Utilities'' tab.

\textbf{IDFEditor:~} The IDFEditor is the simple editor that is distributed with EnergyPlus.~ As an interface, it is crude.~ However, it gets the job done.~ It uses the IDD and then reads and/or creates an IDF file.~ The objects are shown in the groups (see \textbackslash{}group discussion below) and, when an existing file is used, will display how many of an object is found in the IDF.~ This program is installed in the Parent \textbackslash{} PreProcess \textbackslash{} IDFEditor folder.

\textbf{Sample Files:}~ The sample files include several IDF files along with the files the installed version of EnergyPlus created using these files.~ There are several possible child folders here, including the Misc child folder that will contain all the development sample files -- but without having been run for the install.~ These files are installed in the Parent \textbackslash{} ExampleFiles folder and any appropriate child folders under that.

\textbf{Weather Converter:}~ The WeatherConverter program can process raw weather data in several formats into the EnergyPlus weather data format (epw).~ In addition, the WeatherConverter program can be used to generate a simple report of the weather data as well as produce a .csv file of the format.~ The .csv (comma separated variable) format can easily be imported into spread sheet programs or other tables.~ This program is installed in the Parent \textbackslash{} PreProcess \textbackslash{} WeatherConverter folder.~ The WeatherConverter also has a DLL file which could be used directly by an external interface.

\textbf{BLAST Translator:~} The BLAST translator program can be used to convert a BLAST input file into a format that can be executed from EnergyPlus.~ Extensive system translation is not done with this program -- mostly geometry and other space gain elements as well as zone oriented (i.e.~People) schedules.~ If the BLAST input file contains thermostatic controls in the zones, then the EnergyPlus IDF file will include a purchased-air solution of that BLAST input file.~ This program is no longer part of the EnergyPlus install but is available with each new release.

\textbf{DOE-2 Translator:}~ The DOE-2 translator program is similar to the BLAST Translator program but for DOE-2 files.~ DOE-2 translator output must be processed by the EPMacro program prior to running in EnergyPlus (the EP-Launch program/EPL-Run procedure does this automatically for all ``.imf'' files).~ This program is no longer part of the EnergyPlus install but is available with each new release.

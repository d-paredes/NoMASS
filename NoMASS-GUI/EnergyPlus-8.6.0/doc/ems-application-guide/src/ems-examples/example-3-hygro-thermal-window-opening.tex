\section{Example 3. Hygro-thermal Window Opening Control for Airflow Network}\label{example-3.-hygro-thermal-window-opening-control-for-airflow-network}

\subsection{Problem Statement}\label{problem-statement-005}

A user of EnergyPlus version 3.1 posted the following question on the Yahoo! list (circa spring 2009):

I am currently trying to model a simple ventilation system based on an

exhaust fan and outdoor air variable aperture paths that open according to

the indoor relative humidity.

As I didn't find any object to directly do this, I am trying to use an

AirflowNetwork: MultiZone: Component: DetailedOpening object and its

AirflowNetwork: multizone: Surface object to model the variable aperture. But

the Ventilation Control Mode of the surface object can only be done via

Temperature or Enthalpy controls (or other not interesting for my purpose),

and not via humidity.

So my questions are:

1- is it possible to make the surface object variable according to the

relative humidity? (maybe adapting the program?)

2- or is there an other way to make it?

Because the traditional EnergyPlus controls for window openings do not support humidity-based controls (or did not as of Version 3.1), the correct response to Question \#1 was ``No.''~ But with the EMS, we can now answer Question \#2 as ``Yes.''~ How can we take the example file called HybridVentilationControl.idf and implement humidity-based control for a detailed opening in the airflow network model?

\subsection{EMS Design Discussion}\label{ems-design-discussion-005}

The main EMS sensor will be the zone air humidity, so we use an EnergyManagementSystem:Sensor object that maps to the output variable called System Node Relative Humidity for the zone's air node. This zone has the detailed opening.

The EMS will actuate the opening in an airflow network that is defined by the input object AirflowNetwork:MultiZone:Component:DetailedOpening. The program will setup the actuator for this internally, but we need to use an EnergyManagementSystem:Actuator object to declare that we want to use the actuator and provide the variable name we want for the Erl programs.

Because we do not know the exactly what the user had in mind, for this example we assume that the desired behavior for the opening area is that the opening should vary linearly with room air relative humidity. When the humidity increases, we want the opening to be larger. When the humidity decreases, we want the opening to be smaller. For relative humidity below 25\% we close the opening. At 60\% or higher relative humidity, the opening should be completely open. We formulate a model equation for opening factor as

\begin{equation}
F_{open} = \begin{array}{ll}
    0.0 & RH < 25\% \\
    \frac{RH-25}{60-25} & 25\% \leq RH \leq 60\% \\
    1.0 & RH > 60\%  
  \end{array}
\end{equation}

\subsection{EMS Input Objects}\label{ems-input-objects-005}

EMS-related input objects to solve this problem are listed below and are included in the example file called ``EMSAirflowNetworkOpeningControlByHumidity.idf.''

\begin{lstlisting}

EnergyManagementSystem:Sensor,
    ZoneRH , ! Name
    Zone 1 Node, ! Output:Variable or Output:Meter Index Key Name
    System Node Relative Humidity; ! Output:Variable or Output:Meter Name


  EnergyManagementSystem:Actuator,
    MyOpenFactor,                            ! Name
    Zn001:Wall001:Win001,                  ! Component Name
    AirFlow Network Window/Door Opening, ! Component Type
    Venting Opening Factor;    ! Control Type


  EnergyManagementSystem:ProgramCallingManager,
    RH Controlled Open Factor ,    ! Name
    BeginTimestepBeforePredictor , ! EnergyPlus Model Calling Point
    RH_OpeningController ;         ! Program Name 1


  EnergyManagementSystem:Program,
    RH_OpeningController ,     ! Name
    IF ZoneRH < 25,
      SET MyOpenFactor = 0.0 ,
    ELSEIF ZoneRH > 60,
      SET MyOpenFactor = 1.0 ,
    ELSE,
      SET MyOpenFactor = (ZoneRH - 25) / (60 - 25),
    ENDIF;


  Output:EnergyManagementSystem,
    Verbose,
    Verbose,
    Verbose;
\end{lstlisting}

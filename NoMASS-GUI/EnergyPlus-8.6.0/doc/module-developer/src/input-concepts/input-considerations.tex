\section{Input Considerations}\label{input-considerations}

The IDD/IDF concept allows the module developer much flexibility.~ Along with this flexibility comes a responsibility to the overall development of EnergyPlus. Developers must take care not to obstruct other developers with their additions or changes.~ Major changes in the IDD require collaboration among the developers (both module and interface).

In many cases, the developer may be creating a new model -- a new HVAC component, for instance. Then the most straightforward approach is to create a new object class in the IDD with its own unique, self-contained input. This will seldom impact other developers.

In some cases, the developer may be adding a calculation within an existing module or for an existing class of objects. This calculation may require new or different input fields. Then the developer has a number of choices. This section will present some ideas for adding to the IDD that will minimize impact to other developers.

For example, consider the implementation of Other Side Coefficients (OSC) in the IDD.~ Other side coefficients are a simplification for the surface heat balance and were used mostly in BLAST 2.0 before we had interzone surfaces.~ We have carried this forward into EnergyPlus for those users that understand and can use it.~ We'll use it as an example of approaches to adding data items to the IDD.~ Moreover, we'll try to give some hints on which approaches might be used for future additions.

So, you're adding something to EnergyPlus and it is part of an existing module or object class.~ What do you do with your required inputs to your model?~ There are at least four options:

\begin{itemize}
\item
  Embed your values in a current object class definition.
\item
  Put something in the current definition that will trigger a ``GetInput'' for your values.
\item
  Put something in the current definition that will signal a ``special'' case and embed a name (of your item) in the definition (this adds 1 or 2 properties to the object).
\item
  Just get your input and have each of those inputs reference a named object.
\end{itemize}

For example, using the OSC option in surfaces, in the beta 2 version of EnergyPlus we had

~ A8 , \textbackslash{}field Exterior environment

~~~~~~ \textbackslash{}type alpha

~~~~~~ \textbackslash{}note \textless{}for Interzone Surface:Adjacent surface name\textgreater{}

~~~~~~ \textbackslash{}note For non-interzone surfaces enter:

~~~~~~ \textbackslash{}note ExteriorEnvironment, Ground, or OtherSideCoeff

~~~~~~ \textbackslash{}note OSC won't use CTFs

~ N24, \textbackslash{}field User selected Constant Temperature

~ N25, \textbackslash{}field Coefficient modifying the user selected constant

~~~~~~~~~~~~~ temperature

~ N26, \textbackslash{}field Coefficient modifying the external dry bulb temperature

~ N27, \textbackslash{}field Coefficient modifying the ground temperature

~ N28, \textbackslash{}field Combined convective/radiative film coefficient

~~~~~~ \textbackslash{}note if = 0, use other coefficients

~ N29, \textbackslash{}field Coefficient modifying the wind speed term (s/m)

~ N30, \textbackslash{}field Coefficient modifying the zone air temperature part of

~~~~~~~~~~~~~ the equation

\textbf{1)~~}We have done option 1: embed the values in the input.~ (We have also embedded these values in each and every surface derived type (internal data structure) but that can be discussed elsewhere).

When to use:~ It makes sense to embed these values when each and every object (SURFACE) needs these values (e.g.~we need to specify Vertices for Every Surface -- so these clearly should be embedded).

After beta 2, the definition of Surfaces was changed. Obviously option 1 was not a good choice for the OSC data: the data would be rarely used. Our other options were:

\textbf{2)~~}Obviously the ExteriorEnvironment field will remain (but its name was changed to Outside Face Environment).

However, we do not want to embed the values for OtherSideCoef in the Surface items.~ So, if the ExteriorEnvironment continues to reference OtherSideCoef, we can easily trigger a ``GetInput'' for them.~ An additional object class would be necessary for this case.

OtherSideCoef,~ A1, \textbackslash{}field name of OtherSideCoef,

~ A2, \textbackslash{}field SurfaceName (reference to surface using OSC)

~ \ldots{}.

When to use:~ This option can be used for many cases.~ The same object definition will work for option 4 below.~ Obviously, if there is not a convenient trigger in SURFACE but you want to add a feature, this would let you do it without embedding it in the Surface Definition.~ If there is a trigger, such as exists with the ExteriorEnvironment, the A2 field might not be needed.~ This approach would become a bit cumbersome if you expected there to be a lot of these or if there were a one-to-many relationship (i.e.~a single set of OSCs could be used for many surfaces).~ Nevertheless, the approach provides a convenient ``data check''/cross reference that can be validated inside the code.

\textbf{3)~~}We could also have the SURFACE definition reference an OSC name (in this instance).

So, we'd add a field to the Surface that would be the name in the OtherSideCoef object above.~ Then, the OtherSideCoef objects wouldn't need a Surface Name. This is the most straightforward approach: including data in one object by referencing another and it was the approach chosen for the redefined Surface class.

When to use:~ when there is a set of parameters that would be used extensively, then this would provide a name for those.~ If hand editing, then you only would need to change one set of these parameters rather than having to go through many.~ Of course, the OtherSideCoef object wouldn't also have to have the true numbers but could reference yet a third named object\ldots{}\ldots{} (starting to get messy).

\textbf{4)~~}We could have the OtherSideCoef object as above and just ``get'' it as a matter of course.~ (e.g., in the case where we don't have a convenient trigger such as ExteriorEnvironment).

When to use:~ Note that the same structure for 2 works here too.~ It's just not triggered (to get the input) by a value in the other object (SURFACE).

\textbf{Summary}

There are several approaches to adding items to the IDD.~ Developers need to consider impacts to other developers and users early in the implementation planning.

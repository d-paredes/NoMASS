\section{The Basement idd}\label{the-basement-idd}

The basement idd objects and fields are shown below. These objects also appear in the main Energy+.IDD file with the prefix ``GroundHeatTransfer:Basement:''

\begin{lstlisting}
! Basement foundation heat transfer module Input Data Dictionary file
! Created:    August 18, 2000
! Written by: Edward Clements
! Modified for EnergyPlus Auxiliary Application by C. O. Pedersen 10/04
! Description format: FORTRAN var.: description, units, typ. values
\end{lstlisting}

\begin{lstlisting}
SimParameters,
      N1,\field F: Multiplier for the ADI solution:
         \note 0<F<1.0,
         \note typically 0.1 (0.5 for high k soil]
      N2; \field IYRS: Maximum number of yearly iterations:
         \note typically 15-30]
\end{lstlisting}

\begin{lstlisting}
MatlProps,
     N1, \field NMAT: Number of materials in this domain
         \maximum 6
     N2, \field Density for Foundation Wall
         \note typical value 2243
         \units kg/m3
     N3, \field density for Floor Slab
         \note typical value 2243
         \units kg/m3
     N4, \field density for Ceiling
         \note typical value 311
         \units kg/m3
     N5, \field density for Soil
         \note typical value 1500
         \units kg/m3
     N6, \field density for Gravel
         \note typical value 2000
         \units kg/m3
     N7, \field density for Wood
         \note typical value 449
         \units kg/m3
     N8,  \field Specific heat for foundation wall
          \note typical value 880
          \units J/kg-K
     N9,  \field Specific heat for floor slab
          \note typical value 880
          \units J/kg-K
     N10, \field Specific heat for ceiling
          \note typical value 1530
          \units J/kg-K
     N11, \field Specific heat for soil
          \note typical value 840
          \units J/kg-K
     N12, \field Specific heat for gravel
          \note typical value 720
          \units J/kg-K
     N13, \field Specific heat for wood
          \note typical value 1530
          \units J/kg-K
     N14, \field Thermal conductivity for foundation wall
          \note typical value 1.4
          \units W/m-K
     N15, \field Thermal conductivity for floor slab
          \note typical value 1.4
          \units W/m-K
     N16, \field Thermal conductivity for ceiling
          \note typical value 0.09
          \units W/m-K
     N17,  \field thermal conductivity for soil
          \note typical value 1.1
          \units W/m-K
     N18, \field thermal conductivity for gravel
          \note typical value 1.9
          \units W/m-K
     N19; \field thermal conductivity for wood
          \note typical value 0.12
          \units W/m-K
\end{lstlisting}

\begin{lstlisting}
Insulation,
     N1, \field REXT: R Value of any exterior insulation, K/(W/m2)]
         \units m2-K/W
     A1; \field INSFULL: Flag: Is the wall fully insulated?
         \note  True for full insulation
         \note  False for insulation half way down side wall from grade line
\end{lstlisting}

\begin{lstlisting}
SurfaceProps,
     N1, \field ALBEDO: Surface albedo for No snow conditions
         \note typical value 0.16
     N2, \field ALBEDO: Surface albedo for snow conditions
         \note typical value 0.40
     N3, \field EPSLN: Surface emissivity No Snow
         \note typical value 0.94
     N4, \field EPSLN: Surface emissivity  with Snow
         \note typical value 0.86
     N5, \field VEGHT: Surface roughness No snow conditions,cm
         \note typical value 6.0
         \units cm
     N6, \field VEGHT: Surface roughness Snow conditions, cm, ]
         \note typical value 0.25
         \units cm
     A1; \field PET: Flag, Potential evapotranspiration on? T/F]
         \note  Typically, PET is True
\end{lstlisting}

\begin{lstlisting}
BldgData,
     N1, \field DWALL: Wall thickness,
         \note typical value .2]
         \units m
     N2, \field DSLAB: Floor slab thickness,
         \units m
         \maximum 0.25
     N3, \field DGRAVXY: Width of gravel pit beside basement wall
         \units  m
     N4, \field DGRAVZN: Gravel depth extending above the floor slab
         \units m
     N5; \field DGRAVZP: Gravel depth below the floor slab,
         \units m
         \note typical value 0.1
\end{lstlisting}

\begin{lstlisting}
Interior,
   A1, \field COND: Flag: Is the basement conditioned?
       \note TRUE or FALSE
       \note for EnergyPlus this should be TRUE
   N1, \field HIN: Downward convection only heat transfer coefficient
       \units W/m2-K
   N2, \field HIN: Upward convection only heat transfer coefficient
       \units W/m2-K
   N3, \field HIN: Horizontal convection only heat transfer coefficient
       \units W/m2-K
   N4, \field HIN: Downward combined (convection and radiation) heat transfer coefficient
       \units W/m2-K
   N5, \field HIN: Upward combined (convection and radiation) heat transfer coefficient
       \units W/m2-K
   N6; \field HIN: Horizontal combined (convection and radiation) heat transfer coefficient
       \units W/m2-K
\end{lstlisting}

\begin{lstlisting}
ComBldg,
\memo ComBldg contains the monthly average temperatures (C) and possibility of daily variation amplitude
   N1,  \field January average temperature
        \units C
   N2,  \field February average temperature
        \units C
   N3,  \field March average temperature
        \units C
   N4,  \field April average temperature
        \units C
   N5,  \field May average temperature
        \units C
   N6,  \field June average temperature
        \units C
   N7,  \field July average temperature
        \units C
   N8,  \field August average temperature
        \units C
   N9,  \field September average temperature
        \units C
   N10, \field October average temperature
        \units C
   N11, \field November average temperature
        \units C
   N12, \field December average temperature
        \units C
   N13; \field Daily variation sine wave amplitude
        \units C
        \note (Normally zero, just for checking)
\end{lstlisting}

\begin{lstlisting}
EquivSlab,  !  Supplies the EquivSizing Flag
    ! Using an equivalent slab allows non-rectangular shapes to be
    !    modeled accurately.
    ! The simulation default should be EquivSizing = True
   N1, \field APRatio: The area to perimeter ratio for this slab
       \ units m
   A1; \field EquivSizing: Flag
         \note Will the dimensions of an equivalent slab be calculated (TRUE)
         \note or will the dimensions be input directly? (FALSE)]
         \note Only advanced special simulations should use FALSE.
\end{lstlisting}

\begin{lstlisting}
EquivAutoGrid,
     \memo EquivAutoGrid necessary when EquivSizing = TRUE, TRUE is is the normal case.
       N1, \field CLEARANCE: Distance from outside of wall to edge of 3-D ground domain
         \units m
         \note typical value 15m
       N2, \field SlabDepth: Thickness of the floor slab
         \units m
         \note typical value 0.1m
       N3; \field BaseDepth: Depth of the basement wall below grade
         \units m
\end{lstlisting}

\begin{lstlisting}
!
! ******** The following input objects are required only for special cases.
!
\end{lstlisting}

\begin{lstlisting}
AutoGrid,   ! NOTE: AutoGrid only necessary when EquivSizing is false
! If the modelled building is not a rectangle or square, Equivalent
! sizing MUST be used to get accurate results
       N1, \field CLEARANCE: Distance from outside of wall to edge, 15m]
       N2, \field SLABX: X dimension of the building slab, 0-60.0 m]
       N3, \field SLABY: Y dimension of the building slab, 0-60.0 m]
       N4, \field ConcAGHeight: Height of the fndn wall above grade, m]
       N5, \field SlabDepth: Thickness of the floor slab, m, 0.1]
       N6; \field BaseDepth: Depth of the basement wall below grade, m]
\end{lstlisting}

\begin{lstlisting}
ManualGrid, ! NOTE: Manual Grid only necessary using manual gridding
!  (not recommended)
       N1, \field NX: Number of cells in the X direction: 20]
       N2, \field NY: Number of cells in the Y direction: 20]
       N3, \field NZAG: Number of cells in the Z direction
!              above grade: 4 Always]
       N4, \field NZBG: Number of cells in Z dir. below grade: 10-35]
       N5, \field IBASE: X direction cell indicator of slab edge: 5-20]
       N6, \field JBASE: Y direction cell indicator of slab edge: 5-20]
       N7; \field KBASE: Z direction cell indicator
!              of the top of the floor slab: 5-20]
\end{lstlisting}

\begin{lstlisting}
XFACE,       ! NOTE: This is only needed when using manual gridding
!  (not recommended)
!       [XFACE: X Direction cell face coordinates: m]
       N1, N2, N3, N4, N5, N6, N7, N8, N9, N10, N11, N12, N13, N14,
       N15, N16, N17, N18, N19, N20, N21, N22, N23, N24, N25, N26,
       N27, N28, N29, N30, N31, N32, N33, N34, N35, N36, N37, N38,
       N39, N40, N41, N42, N43, N44;
\end{lstlisting}

\begin{lstlisting}
YFACE,  !NOTE: This is only needed when using manual gridding
!  (not recommended)
!       [YFACE: Y Direction cell face coordinates: m],
       N1, N2, N3, N4, N5, N6, N7, N8, N9, N10, N11, N12, N13, N14,
       N15, N16, N17, N18, N19, N20, N21, N22, N23, N24, N25, N26,
       N27, N28, N29, N30, N31, N32, N33, N34, N35, N36, N37, N38,
       N39, N40, N41, N42, N43, N44;
\end{lstlisting}

\begin{lstlisting}
ZFACE,  !NOTE: This is only needed when using manual gridding
!  (not recommended)
!       [ZFACE: Z Direction cell face coordinates: m]
       N1, N2, N3, N4, N5, N6, N7, N8, N9, N10, N11, N12, N13, N14,
       N15, N16, N17, N18, N19, N20, N21, N22, N23, N24, N25, N26,
       N27, N28, N29, N30, N31, N32, N33, N34, N35, N36, N37, N38,
      N39, N40;
\end{lstlisting}

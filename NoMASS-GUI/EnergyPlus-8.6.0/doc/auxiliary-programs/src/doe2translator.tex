\chapter{DOE2Translator}\label{doe2translator}

The DOE2Translator program creates an EnergyPlus input file from a DOE-2.1E input file. The translation is not intended to be complete but simply an aid to help you move your library of DOE-2.1E input files into EnergyPlus. You should look over the resulting EnergyPlus input file, review the documentation of EnergyPlus, and make any necessary edits to the translated file. 

To download the latest version of the DOE2Translator, go to \href{https://energyplus.net/extras}{energyplus.net/extras} and look for the Additional Release Components section.


To use the DOE2Translator program, at the DOS prompt (or the command prompt for Windows NT/2000 systems), go to the directory that the DOE2Translator is located. That directory is likely to be:

c:\textbackslash{}EnergyPlus\textbackslash{}PreProcess\textbackslash{}DOE2Translator

In this directory there should be several files:

DOE2Translator.exe - the program

D2E\_macr.txt - a support file that contains the header of the translated file

D2comkey.dat - a support file that contains a list of DOE-2 keywords

Samp4n.inp - sample DOE-2.1E input file

Samp4n.imf - the EnergyPlus macro input file resulting from the sample

To use the DOE2Translator simply type

DOE2Translator \textless{}file\textgreater{}

Where you substitute the file you want to translate for \textless{}file\textgreater{} without a file extension. The ``.inp'' file extension is assumed. For example, if you want to translate one of the sample DOE-2.1E input files you would type:

DOE2Translator samp1b

The \textless{}file\textgreater{} can also have a full path, but it should not have an extension. If you have spaces in your path name, enclose the whole thing in ``.

Several files get created when you run the DOE2Translator program. In the same directory as the DOE-2.1E input file, a new file with the same name and the file extension of ``.imf'' contains the EnergyPlus translation.

This is an EnergyPlus macro file and the macro processor EPMacro needs to be used. The DOE2Tranlator uses many macros so using EPMacro is essensial. EP-Launch automatically runs EP-Macro when an ``.imf'' file is selected. In the translated file, comments shown with a tilde ``\textasciitilde{}'' are messages from the DOE2Translator program that may be helpful in understanding the limits of the translation.

The D2EP.log file contains a detailed log of the translation process. The D2E\_TEMP.txt file contains an intermediate version of the log file. Both of these files are created in the same directory as the DOE2Translator program and can usually be deleted.

Since DOE-2.1e and EnergyPlus share a common macro language, many macro features are passed to the EnergyPlus file unchanged, including \#\#set1, \#\#if, \#\#def and other macro commands. References to macro variables (i.e., ``var{[}{]}'') and expressions (i.e., " \#{[}x{[}{]} + 1{]}``) are usually passed through to the resulting EneryPlus IMF unless the DOE2Translator needs to understand that field during the translation process. The DOE2Translator does not evaluate macro expressions and if it needs to understand a field value and a macro is present instead will use a default value for the field instead. Most fields do not need to be understood by the translator and are directly passed through to the IMF file unchanged.

Files that are included with the \#\#include are not translated automatically and would each need to be run through the DOE2Translator. If the included file does not have the INP extension it would need to be changed prior to the translation. In addition, the user would need to edit the \#\#include commands to use the IMF extension instead of the INP extension.

In this version of the DOE2Translator program, translation is limited to the following DOE-2 commands, which represent the majority of the building envelope and internal gains:

\begin{lstlisting}
    SPACE (except SHAPE = BOX)
    SPACE-CONDITIONS
    DAY-SCHEDULE (except use of HOURS and VALUES keywords)
    WEEK-SCHEDULE (except use of DAYS and DAY-SCHEDULE keywords)
    SCHEDULE (except use of WEEK-SCHEDULE keyword)
    MATERIAL
    LAYERS
    CONSTRUCTION
    EXTERIOR-WALL, ROOF (except polygon)
    INTERIOR-WALL
    FIXED-SHADE
    WINDOW
    DOOR
    RUN-PERIOD
    DESIGN-DAY
    LIKE
    SET-DEFAULT
\end{lstlisting}
